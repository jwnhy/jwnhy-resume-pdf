\documentclass{resume}
\usepackage[paperwidth=210mm,paperheight=400mm,margin=20mm]{geometry}
\usepackage[svgnames]{xcolor}
\usepackage{hyperref}
\hypersetup{%
  colorlinks=true,          % hyperlinks
  allcolors=DarkRed,%
  pdfborderstyle={/S/U/W 1} % border style will be underline of width 1pt
}

\usepackage{vwcol}
\usepackage[numbers]{natbib}
\usepackage{cfr-lm}         % make old style numbers default
\usepackage{graphicx}

\usepackage[T1]{fontenc}
\usepackage{tgtermes}

\usepackage{atbegshi}


\begin{document}

%%%%%%%%%%%%%%%%%%%%%%%%%%%%%%%%%%%%%%%%%%%%%%%% TITLE SECTION
\begin{center}
	\name{LU Hongyi}
	\contact
	{+86 156 2529 0103}
	{\href{mailto://luhy2017@mail.sustech.edu.cn}{luhy2017@mail.sustech.edu.cn}}
	{\href{https://hongyi.lu}{https://hongyi.lu}}
\end{center}

%%%%%%%%%%%%%%%%%%%%%%%%%%%%%%%%%%%%%%%%%%%%%%%% EDUCATION 
\section{Education}
\begin{content}
	{\bf HKUST}
	\hfill {\bf 2022
		-- Present}
	\\ {Ph.D. Computer Science}

	{\bf SUSTech} \hfill {\bf 2017 --
		2021} \\
	{B.Sc. Mathematics} \hfill

	% \vspace{-.25\baselineskip}
	% {\em Honours:} If you'd rather list your awards here than in a separate section later, simply write the awards here and comment out that section (and vice versa).
	\sectionlineskip
\end{content}

%%%%%%%%%%%%%%%%%%%%%%%%%%%%%%%%%%%%%%%%%%%%%%%% RESEARCH EXPERIENCE
\section{Working Experience}
\begin{content}

	\begin{position}{COMPASS Lab.}{Nov. 2021 -- 2022}{Research
			Assistant}{Prof.~Zhang Fengwei}{Southern University of Science and
			Technology}
		\item Conducted research in computer systems.
	\end{position}

	\sectionlineskip
\end{content}
\vspace{-3\medskipamount}

%%%%%%%%%%%%%%%%%%%%%%%%%%%%%%%%%%%%%%%%%%%%%%%%% PUBLICATIONS 
\noindent
\renewcommand{\refname}{Publications}   % name of the reference section
\bibliography{biblio.bib}               % bib file
\bibliographystyle{plainyr-rev}            % citation style
\nocite{*}
\sectionlineskip

In total, I have published \textbf{four CCF-A papers} as first/co-first author~(marked with *), \textbf{two in Security Big4 conferences}.


\section{Academic Service}
\begin{content}
  {\bf Official Reviewer} \enspace TIFS, TDSC

		{\bf Ext. Reviewer} \enspace USENIX Security 2023, S\&P 2023, PoPET 2023, CCS 2023, CCS 2022
\end{content}

%%%%%%%%%%%%%%%%%%%%%%%%%%%%%%%%%%%%%%%%%%%%%%%%% PROJECTS
\section{Projects}

\begin{content}
  {\bf \textbf{\textsc{CuSafe}}} \enspace
  {\href{https://sites.google.com/view/safe-gpu/}{\texttt{sites.google.com/view/safe-gpu}}}

  {We built a system named \textsc{CuSafe}. It detects memory safety vulnerabilities in CUDA programs.}

  {Our evaluation shows that \textsc{CuSafe} beats existing tools such as compute-sanitizer~(\textbf{NVIDIA}), cuCatch~(PLDI '22, \textbf{NVIDIA}), and LMI~(HPCA' 25) in both performance and accuracy; \textsc{CuSafe} also supports the LLMs such as LLaMA2 and LLaMA3. \textsc{CuSafe} is approx. 13$\times$ faster than the compute-sanitizer, which is the official tool provided by \textbf{NVIDIA}.}

  {I proposed this project and worked as the sole student; I wrote all the code myself.}
  \hfill {\bf 2025}


  {\bf \textbf{\textsc{Mole}} (CCS' 25)} \enspace
  {\href{https://sites.google.com/view/mole-gpu}{\texttt{sites.google.com/view/mole-gpu}}}

  {We proposed a new attack against existing GPU TEEs; it leverages an under-documented MCU in Arm Mali GPUs to break the security of GPU TEEs. \textbf{Arm} have acknowledged our findings and enhanced their supply-chain security.}

  {It breaks multiple GPU TEEs, such as StrongBox~(CCS' 22, \textbf{Ant Group}), CAGE~(NDSS' 25, \textbf{Ant Group}) and MyTEE~(NDSS' 23)}

  {I proposed this project and worked as the leader; I wrote the most part of the code myself.}
  \hfill{\bf 2025}


  {\bf \textbf{\textsc{Moat}} (USENIX Security' 24)} \enspace
  {\href{https://sites.google.com/view/safe-bpf/}{\texttt{sites.google.com/view/safe-bpf}}
    
    {We build a system named \textsc{Moat}. It isolates BPF programs using hardware features like Intel MPK/Arm Stage-II.}

    {We have worked with a company to put it into production for \textbf{1,170,000 Yuan}.}
    
    {I proposed this project and worked as the leader; I wrote all the code myself.}
    \hfill {\bf 2024}


	{\bf LLVM} \enspace
	{\href{https://github.com/llvm/}{\texttt{github.com/llvm}}

		{I submitted a patch to the LLVM project, fixing a bug of its debugger.} }
	\hfill {\bf 2023}

	{\bf Patent ZL 2022 1 0436969.8}

	{Kernel Space Debug Method, Device and Storage Medium}
	\hfill {\bf 2022}

	{\bf Translation of Software Foundation} \enspace
	{\href{https://coq-zh.github.io/SF-zh/}{\texttt{coq-zh.github.io/SF-zh}}

		{I helped translate the Software Foundation, a famous textbook
			about formal verification.}}
	\hfill {\bf 2022}

  {\bf \textbf{\textsc{BadUSB-C}} (WOOT' 21)}

  {We uncovered a novel attack vector against USB-C devices. It enhances the power of \textsc{BadUSB} attacks with the video capability of USB-C. We received \textbf{30,000 Yuan} bounty award from the affected vendor.}
  \hfill {\bf 2021}


	\sectionlineskip
\end{content}

\end{document}
