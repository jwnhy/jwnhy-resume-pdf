% !TEX program = xelatex
\documentclass{resume}
\usepackage[paperwidth=210mm,paperheight=500mm,margin=20mm]{geometry}
\usepackage[svgnames]{xcolor}
\usepackage{hyperref}
\hypersetup{%
  colorlinks=true,          % hyperlinks
  allcolors=DarkRed,%
  pdfborderstyle={/S/U/W 1} % border style will be underline of width 1pt
}


\usepackage{vwcol}
\usepackage[numbers]{natbib}
\usepackage{cfr-lm}         % make old style numbers default
\usepackage{graphicx}

\usepackage[T1]{fontenc}
\usepackage{tgtermes}

\usepackage{atbegshi}

\usepackage{ctex}
\begin{document}

%%%%%%%%%%%%%%%%%%%%%%%%%%%%%%%%%%%%%%%%%%%%%%%% TITLE SECTION
\begin{center}
	\name{卢 弘毅}
	\contact
	{+86 156 2529 0103}
	{\href{mailto://luhy2017@mail.sustech.edu.cn}{luhy2017@mail.sustech.edu.cn}}
	{\href{https://hongyi.lu}{https://hongyi.lu}}
\end{center}

%%%%%%%%%%%%%%%%%%%%%%%%%%%%%%%%%%%%%%%%%%%%%%%% EDUCATION 
\section{\textbf{教育背景}}
\begin{content}
	{\textbf{香港科技大学}}
	\hfill {\textbf{2022 -- 至今}}
	\\ \textbf{计算机科学博士}

	\textbf{南方科技大学} \hfill \textbf{2017 -- 2021} \\
	\textbf{数学学士} \hfill

	% \vspace{-.25\baselineskip}
	% {\em Honours:} If you'd rather list your awards here than in a separate section later, simply write the awards here and comment out that section (and vice versa).
	\sectionlineskip
\end{content}

%%%%%%%%%%%%%%%%%%%%%%%%%%%%%%%%%%%%%%%%%%%%%%%% RESEARCH EXPERIENCE
\section{\textbf{工作经历}}
\begin{content}

	\begin{position}{COMPASS Lab.}{Nov. 2021 -- 2022}{科研助理}{张锋巍教授}{南方科技大学}
		\item 完成计算机系统相关的研究工作。
	\end{position}

	\sectionlineskip
\end{content}
\vspace{-3\medskipamount}

%%%%%%%%%%%%%%%%%%%%%%%%%%%%%%%%%%%%%%%%%%%%%%%%% PUBLICATIONS 
\noindent
\renewcommand{\refname}{\textbf{学术发表}}   % name of the reference section
\bibliography{biblio.bib}               % bib file
\bibliographystyle{plainyr-rev}            % citation style
\nocite{*}
\sectionlineskip

我合计作为第一作者/共同第一作者(以*标记)发表\textbf{四篇CCF-A类论文},其中两篇是安全领域的“四大”会议。
另外,我还有一篇在投的安全“四大”论文与一篇正在大修的CCF-A类期刊论文。


\section{\textbf{学术服务}}
\begin{content}
	\textbf{正式审稿人} \enspace TIFS, TDSC

	\textbf{外部审稿人} \enspace USENIX Security 2023, S\&P 2023, PoPET 2023, CCS 2023, CCS 2022
\end{content}

%%%%%%%%%%%%%%%%%%%%%%%%%%%%%%%%%%%%%%%%%%%%%%%%% PROJECTS
\section{\textbf{项目介绍}}

\begin{content}
	\textbf{论文项目:}\textsc{CuSafe} \enspace
	{\href{https://sites.google.com/view/safe-gpu/}{\texttt{sites.google.com/view/safe-gpu}}}

	{我们创建了一个名为\textsc{CuSafe}的系统。 该系统支持对CUDA程序中的内存安全问题进行检测。}

	{我们的评测显示\textsc{CuSafe}在性能和准确率两方面都超越了已有的相关工具,例如compute-sanitizer(由\textbf{NVIDIA}开发),cuCatch(PLDI’ 22,由\textbf{NVIDIA}开发)与LMI(HPCA’ 25)}

	{我提出了该项目并作为唯一的学生负责该项目,我独自完成了整个项目的开发。}
	\hfill {\bf 2025}


	\textbf{论文项目:}\textsc{Mole} (CCS' 25) \enspace
	{\href{https://sites.google.com/view/mole-gpu}{\texttt{sites.google.com/view/mole-gpu}}}

	{我们提出了一种针对现有GPU可信执行环境的新式攻击。该攻击利用了一个在Arm GPU中的隐藏MCU来破坏GPU可信执行环境的安全性。}

	{该攻击攻破了多个GPU可信执行环境,例如StrongBox(CCS’ 22,由\textbf{蚂蚁集团}开发)、CAGE(NDSS’ 24,由\textbf{蚂蚁集团}开发)和MyTEE(NDSS’ 23)。}

	{我提出了该项目并作为项目负责人,我开发了该项目的大部分代码,另一位同学邓韵杰完成了OpenCL用户态攻击组件与攻击测试程序的开发。}
	\hfill{\bf 2025}


	\textbf{论文项目:}\textsc{Moat} (USENIX Security' 24) \enspace
	{\href{https://sites.google.com/view/safe-bpf/}{\texttt{sites.google.com/view/safe-bpf}}}

	{我们创建了一个名为\textsc{Moat}的系统。该系统能够利用Intel MPK/Arm StageII等硬件特性隔离恶意BPF程序以阻止对它们内核的攻击}

	{我们正在与企业合作将该项目推向生产环境,该项目提供了\textbf{117万人民币的资助}。}

	{我提出了该项目并作为项目负责人,我独自完成了整个项目的开发。目前与企业的合作由我负责指导,另一位同学黄沥剑负责开发。}
	\hfill {\bf 2024}

	\textbf{发明专利 2022 1 0436969.8}

	{内核空间调试方法及其装置、计算机设备、存储介质}
	\hfill {\bf 2022}

	\textbf{《软件基础》翻译工作} \enspace
	{\href{https://coq-zh.github.io/SF-zh/}{\texttt{coq-zh.github.io/SF-zh}}

		{我协助翻译了《软件基础》中的部分内容。《软件基础》是形式化验证领域的一本知名教科书。}}
	\hfill {\bf 2022}

	\textbf{论文项目:}BadUSB-C (WOOT' 21)

	{我们发现了针对USB-C设备的一种新式攻击,其利用USB-C的视频传输能力增强了传统BadUSB的攻击能力。该项目被受影响企业认定为高危漏洞,我们因此获得了\textbf{3万元人民币的赏金}}
	\hfill {\bf 2021}


	\sectionlineskip
\end{content}

\end{document}
